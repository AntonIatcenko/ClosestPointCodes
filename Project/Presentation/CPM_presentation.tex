\documentclass{beamer}
 
\usepackage[utf8]{inputenc}
\usetheme{Copenhagen}
\usecolortheme[snowy]{owl}
\setbeamersize{text margin left=10pt,text margin right=10pt} 
 
\usepackage{biblatex}
\usepackage{tikz}
\usepackage{enumerate}
\usepackage{scrextend}
\usepackage{amsmath}
\usepackage{amsfonts}
\usepackage{amssymb}
\usepackage{relsize}
\usepackage{dsfont}
\usepackage{graphicx}
\usepackage{multimedia}

\setbeamercolor*{block title}{
    use=normal text,
    fg=normal text.fg,
    bg=normal text.bg!80!normal text.fg
} 

\setbeamercolor*{block body}{
bg=normal text.bg!90!normal text.fg
}
 
 
 
\title{Swift-Hohenberg on a Torus}
\author{Anton Iatcenko}
%\institute{Presentation for Math 381W}
\date{Summer 2017}
 
 
 
\begin{document} 
 
 
 
\frame{\titlepage}


 
 
\begin{frame}
\frametitle{Plan} 

\begin{itemize}

\item Overview of the Closest Point Method.

\item Biharmonic heat equation on a sphere and a torus. 

\item Swift-Hohenberg equation.

\end{itemize}

\end{frame}



\begin{frame}
\frametitle{Closest Point Method} 

Idea: instead of discretizing surface derivatives using parametrization, we represent them by Cartesian derivatives acting
on the extension of a surface quantity to the embedding space. To achieve this, data needs to be constant in normal direction. 


\begin{block}{Gradient Principle}
For points $x$ on a smooth surface $S$, $\nabla_S u(x) = \nabla(u(cp(x)))$ because the function u(cp(x)) is constant in the
normal direction and therefore only varies along the surface. In other words, at points x on the surface, intrinsic surface
gradients $\nabla_S u(x)$ are the same as gradients of u(cp(x)).
\end{block}

Approximation of surface operators via Cartesian grid operators acting on normally extended data...

\end{frame}

 

 
\begin{frame}
\frametitle{Closest Point Extension} 
 
To obtain the normal extension, we recall that the vector connecting a point in the embedding space to its closest point on 
a surface is in fact normal to it.

 
 
 
\end{frame} 


\begin{frame}
\frametitle{Stabilized Diffusion Operator} 


spectra of standard and stabilized operators. 


\end{frame}


\begin{frame}
\frametitle{Stabilized Diffusion Operator} 

spy plots on sphere.

\end{frame}



\begin{frame}
\frametitle{Stability Restrictions }
 
stability restrictions 
 
\end{frame} 
 
 
\begin{frame}
\frametitle{Convergence on the Sphere}
 
\end{frame}  


\begin{frame}
\frametitle{Swift-Hohenberg Equation}
 

\begin{gather*}
u_t = -\triangle^2 u - \triangle + u (P-1)u + su^2 - u^3
\end{gather*} 
 
This equation was derived by Swift and Hohenberg in 1977 to study thermal fluctuations on a fluid near the Rayleigh-Benard
convective instability. The function $u$ is the temperature field in a plane horizontal layer of fluid heated from below. 
The parameter $r$ measures how far the temperature is above the minimum temperature required for convection: for $r<0$,
the heating is too small to cause convection, while for $r>0$, convection occurs. The Swift-Hohenberg equation is an example of 
a PDE that exhibits pattern formation, including stripes, spots and spirals.
 
\end{frame}
 
 
 
\begin{frame}
\frametitle{Discretization}
 
\begin{gather*}
u_t = \underbrace{-\triangle^2 u - \triangle + u (P-1)u}_{=:Lu} + \underbrace{su^2 - u^3}_{=:Nu}
\end{gather*} 
 
 
Linear part is very stiff, but can be inverted (-ish) - treat implicitly.
Nonlinear part - difficult to invert, treat explicitly.  
 
\end{frame} 
 
 
\begin{frame} 
 
\begin{center}
\Huge{Thank you for your attention!}
\end{center} 
 
\end{frame} 
 
 
 
\begin{frame}
\frametitle{Videos} 
 
 
\end{frame}  
 
 
 
 
 
%\begin{frame}
%\frametitle{References}
% 
% 
%\begin{thebibliography}{9}
%\bibitem{implcpm} 
%The Implicit Closest Point Method for the Numerical Solution of Partial Differential Equations on Surfaces.
%\textit{SIAM J. SCI. COMPUT. 2009 Society for Industrial and Applied Mathematics Vol. 31, No. 6, pp. 4330-4350}
% 
%\bibitem{implcpm} 
%The Implicit Closest Point Method for the Numerical Solution of Partial Differential Equations on Surfaces.
%\textit{SIAM J. SCI. COMPUT. 2009 Society for Industrial and Applied Mathematics Vol. 31, No. 6, pp. 4330-4350} 
% 
%
%\end{thebibliography} 
%  
%\end{frame} 
 
 
 
 
 
 
\end{document} 
 
 
 
 
 
 
 
 
 
 
 
 
 
 
  
 
 
 